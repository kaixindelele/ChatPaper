Learning Algorithms}

Several key publications have contributed to the development of reinforcement learning (RL) algorithms for robot control. Mnih et al. \citep{mnih2013playing} introduced the first successful deep RL model, which learns control policies directly from high-dimensional sensory input. Lillicrap et al. \citep{lillicrap2015continuous} proposed an actor-critic algorithm based on deterministic policy gradients that can operate in continuous action spaces. Haarnoja et al. \citep{haarnoja2018soft} presented soft actor-critic, an off-policy RL algorithm that achieves state-of-the-art performance on continuous control tasks. He et al. \citep{he2020reinforcement} developed an RL control strategy based on the actor-critic structure for vibration suppression in a flexible two-link manipulator system. Liu et al. \citep{liu2021deep} addressed the challenges of sample efficiency and generalization in deep RL algorithms for robotic manipulation control.

\paragraph{Benchmarking RL for Robot Control}

Benchmarking RL algorithms in the context of robot control is crucial for evaluating their performance and comparing different approaches. Thrun et al. \citep{thrun2002probabilistic} proposed planning and navigation algorithms that exploit statistics from uncertain real-world environments to guide robots. Nasiriany et al. \citep{nasiriany2021augmenting} introduced Manipulation Primitive-augmented RL (MAPLE), a framework that combines RL algorithms with a library of behavior primitives for manipulation tasks. Parker-Holder et al. \citep{parker-holder2022automated} surveyed the field of automated RL (AutoRL) and provided a taxonomy for different areas of research. Majumdar et al. \citep{majumdar2019a} discussed scalable semidefinite programming approaches for RL, including low-rank approximate solutions and augmented Lagrangian techniques. Zhang et al. \citep{zhang2021learning} proposed an inverse RL approach to recover variable impedance policies and reward functions from expert demonstrations.

\paragraph{Application Domains}

RL has been successfully applied to various domains in robot control. Li et al. \citep{li2021reinforcement} developed a model-free RL framework for training locomotion policies in simulation and transferring them to a real bipedal robot. Kim et al. \citep{kim2021review} categorized machine learning approaches in soft robotics, including soft sensors, actuators, and wearable robots. Katz et al. \citep{katz2019mini} used Convex Model-Predictive Control (cMPC) to generate dynamic gaits on the Mini Cheetah robot. Siekmann et al. \citep{siekmann2021blind} demonstrated sim-to-real RL for robust locomotion over stair-like terrain on the Cassie robot. Wang et al. \citep{wang2021data} proposed a data-driven RL control scheme for unmanned surface vehicles in complex marine environments.

\paragraph{Challenges and Limitations}

Despite the successes, RL in robot control still faces challenges and limitations. Gao et al. \citep{gao2020reinforcement} introduced flexible policy iteration (FPI) to address sample inefficiency and stability in RL controllers. Tran et al. \citep{tran2019safety} proposed a forward reachability analysis approach to verify the safety of cyber-physical systems with RL controllers. Wang et al. \citep{wang2017safety} presented safety barrier certificates for collision-free behaviors in multirobot systems. Liu et al. \citep{liu2021deep} discussed the challenges of sample efficiency and generalization in deep RL algorithms for robotic manipulation control. Margolis et al. \citep{margolis2022rapid} proposed an end-to-end learned controller for the MIT Mini Cheetah robot, highlighting the need for robustness to disturbances.

\paragraph{Future Research Directions}

Several future research directions can further advance RL for robot control. Zhang et al. \citep{zhang2021learning} explored the use of transfer learning in RL for robot control. Yang et al. \citep{yang2020combating} discussed the potential of multi-agent RL in addressing risks and challenges in robotics. Hespanha et al. \citep{hespanha2007a} reviewed estimation, analysis, and controller synthesis for networked control systems. Morgan et al. \citep{morgan2021model} proposed Model Predictive Actor-Critic (MoPAC), a hybrid model-based/model-free RL method. Kober et al. \citep{kober2013reinforcement} provided a comprehensive survey of RL in robotics, highlighting potential future research directions.

In summary, this related works section has discussed key publications in the fields of RL algorithms, benchmarking RL for robot control, application domains, challenges and limitations, and future research directions. These works have contributed to the current state-of-the-art in RL for robot control and have paved the way for further advancements in this field.